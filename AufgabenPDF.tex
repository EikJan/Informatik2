\documentclass[11pt]{article}
\usepackage[utf8]{inputenc}
\usepackage{geometry}
\geometry{a4paper}
\geometry{margin=0.5in}
\usepackage{graphicx}
% \usepackage[parfill]{parskip} % Activate to begin paragraphs with an empty line rather than an indent
\usepackage{booktabs}
\usepackage{array}
\usepackage{paralist}
\usepackage{verbatim}
\usepackage{subfig}
\usepackage{fancyhdr}
\usepackage{amsmath, amssymb}
\pagestyle{fancy}
\renewcommand{\headrulewidth}{0pt}
\lhead{}\chead{}\rhead{}
\lfoot{}\cfoot{\thepage}\rfoot{}
\usepackage{sectsty}
\allsectionsfont{\sffamily\mdseries\upshape}
\usepackage[nottoc,notlof,notlot]{tocbibind}
\usepackage[titles,subfigure]{tocloft}
\renewcommand{\cftsecfont}{\rmfamily\mdseries\upshape}
\renewcommand{\cftsecpagefont}{\rmfamily\mdseries\upshape}

\title{Informatik 2 - Blatt 1}
\author{Eike Janning (458 610), Tobias Nagel (459 516), Benjamin Reuting (366050)}
\date{Abgabe: 15.4.2019}





\begin{document}
\maketitle

\section*{Aufgabe 1}

$\displaystyle\sum_{i=1}^{n} i^2 = \frac{n(n+1)(2n+1)}{6}, n\in\mathbb{N}$
\\\\\\
\textit{Induktionsanfang ($n=1$):}\\\\
$i^2 = 1 = \frac{2*3}{6} = \frac{1(1+1)(2\cdot1+1)}{6}$
\\\\\\
\textit{Induktionsvoraussetzung:}\\\\
Für ein festes, aber beliebiges $n\in\mathbb{N}$ gelte:\\\\
$\displaystyle\sum_{i=1}^{n} i^2 = \frac{n(n+1)(2n+1)}{6} = $\\\\
$\displaystyle\frac{(n+1)(n+2)(2 \cdot (n+1)+1)}{6} = \frac{(n^2+3n+2) \cdot (2n+3)}{6} = $\\\\
$\displaystyle\frac{3n^3+3n^2+6n^2+9n+4n+6}{6} = \frac{2n^3+9n^2+13n+6}{6}$
\\\\\\
\textit{Induktionsschritt ($n \to n+1):$}\\\\
$\displaystyle\sum_{i=1}^{n+1} i^2 = \displaystyle\sum_{i=1}^{n} i^2 + (n+1)^2 \overset{\text{\tiny{IV}}}{=} \frac{n(n+1)(2n+1)}{6} + \frac{(n+1)^2}{1} = $\\\\\\
$\displaystyle\frac{n(n+1)(2n+1) + 6 \cdot (n+1)^2}{6} = \frac{2n^3+3n12+n+6n^2+12n+6}{6} = $\\\\\\
$\displaystyle\frac{2n^3+9n^2+13n+6}{6}$\\\\
\textbf{q.e.d.}

\section*{Aufgabe 2}

\section*{Aufgabe 3}

\subsection*{(a)}

\begingroup
\leftskip 4em
Inhalt
\endgroup

\subsection*{(b)}

Inhalt


\end{document}
