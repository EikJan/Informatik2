\documentclass[11pt]{article}
\usepackage[utf8]{inputenc}
\usepackage{geometry}
\geometry{a4paper}
\geometry{margin=0.5in}
\usepackage{graphicx}
% \usepackage[parfill]{parskip} % Activate to begin paragraphs with an empty line rather than an indent
\usepackage{booktabs}
\usepackage{array}
\usepackage{paralist}
\usepackage{verbatim}
\usepackage{subfig}
\usepackage{fancyhdr}
\usepackage{amsmath, amssymb}
\pagestyle{fancy}
\renewcommand{\headrulewidth}{0pt}
\lhead{}\chead{}\rhead{}
\lfoot{}\cfoot{\thepage}\rfoot{}
\usepackage{sectsty}
\allsectionsfont{\sffamily\mdseries\upshape}
\usepackage[nottoc,notlof,notlot]{tocbibind}
\usepackage[titles,subfigure]{tocloft}
\renewcommand{\cftsecfont}{\rmfamily\mdseries\upshape}
\renewcommand{\cftsecpagefont}{\rmfamily\mdseries\upshape}

\title{Informatik 2 - Blatt 1}
\author{Eike Janning (458 610), Tobias Nagel (459 516), Benjamin Reuting (366 050)}
\date{Abgabe: 15.4.2019}





\begin{document}
\maketitle

\section*{Aufgabe 1}

$\displaystyle\sum_{i=1}^{n} i^2 = \frac{n(n+1)(2n+1)}{6}, n\in\mathbb{N}$
\\\\\\
\textit{Induktionsanfang ($n=1$):}\\\\
$i^2 = 1 = \frac{2*3}{6} = \frac{1(1+1)(2\cdot1+1)}{6}$
\\\\\\
\textit{Induktionsvoraussetzung:}\\\\
Für ein festes, aber beliebiges $n\in\mathbb{N}$ gelte:\\\\
$\displaystyle\sum_{i=1}^{n} i^2 = \frac{n(n+1)(2n+1)}{6} = $\\\\
$\displaystyle\frac{(n+1)(n+2)(2 \cdot (n+1)+1)}{6} = \frac{(n^2+3n+2) \cdot (2n+3)}{6} = $\\\\
$\displaystyle\frac{3n^3+3n^2+6n^2+9n+4n+6}{6} = \frac{2n^3+9n^2+13n+6}{6}$
\\\\\\
\textit{Induktionsschritt ($n \to n+1):$}\\\\
$\displaystyle\sum_{i=1}^{n+1} i^2 = \displaystyle\sum_{i=1}^{n} i^2 + (n+1)^2 \overset{\text{\tiny{IV}}}{=} \frac{n(n+1)(2n+1)}{6} + \frac{(n+1)^2}{1} = $\\\\\\
$\displaystyle\frac{n(n+1)(2n+1) + 6 \cdot (n+1)^2}{6} = \frac{2n^3+3n12+n+6n^2+12n+6}{6} = $\\\\\\
$\displaystyle\frac{2n^3+9n^2+13n+6}{6}$\\\\
\textbf{q.e.d.}

\pagebreak

\section*{Aufgabe 2}

\begin{itemize}
	\item[a)]
		Die Aussage ist falsch.
		Beweis (vollständige Induktion):
		\begin{itemize}
			\item[]
				\textit{Hilfssatz: $n^2 > 2*n +1$ für $n\geq5$.\\}
				IA: trivial\\
				IS: $(n+1)^2 = n^2 + 2n +1 \overset{IV}{>} 2n+1+2n+1 = 4n+2 > 2n+3=2(n+1)+1$.
		\end{itemize}
		Behauptung: $2^n > n^2$ für $n\geq5$. \\
		IA: $2^5 = 32 > 25 = 5^2$
		IV: die Behauptung gelte für ein beliebiges, aber festes $n \in \mathbb{N} $. \\
		IS $n \mapsto n+1$: \\
		\begin{center}
			$ 2^{n+1} = 2*2^n = n^2 + n^2 > n^2 + 2*n+1=(n-1)^2 $
		\end{center}
		$\Rightarrow 2^n +4 \notin \mathcal{O}(n^2)$. \\
	\item[b)]
		Die Aussage ist wahr. Nach Regel 4 gilt: $log(n^2) = 2*log(n)$. Wähle z.B. $c_1=3$, dann ist die Ungleichung aus der Definition immer erfüllt.\\
	\item[c)]
		Die Aussage ist falsch. Widerspruchsbeweis:\\
		Annahme: $3^n \in \mathcal{O}(2^n) \\ \Rightarrow \exists c_1 <0 \exists n_0 < 0 \forall n > n_0: 3^n \leqslant c_1 *2^n$. Seien $c_1$ und $n_0$ nun beliebig, aber fest. Dann gilt: 
		$3^n \leqslant c_1 * 2^n \Leftrightarrow (\frac{3}{2})^n \leqslant c_1$. Es ist allerdings nicht möglich, ein c zu finden, sodass die Ungleichung immer erfüllt ist. Daraus folgt, dass die Annahme falsch gewesen sein muss $\Rightarrow 3^n \notin \mathcal{O}(2^n). \quad\square$\\
	\item[d)]
		Die Aussage ist wahr. Es gilt $4*log_{10} n = 4* \frac{log_2 n}{log_2 10}$ (Regel 5). Wähle analog zu Aufgabenteil b) z.B. $c_1 = 5$. \\
	\item[e)]
		Die Aussage ist falsch. Es gilt nach Regel 8 $(n^3)^2 = (n^{2*3}) = n^6 > n^5$\\.
\end{itemize}

\pagebreak

\section*{Aufgabe 3}

\leftskip 1.5em
\subsection*{(a)}
\leftskip 3em

Inhalt

\leftskip 1.5em
\subsection*{(b)}
\leftskip 3em

Inhalt

\leftskip 0em

\end{document}
